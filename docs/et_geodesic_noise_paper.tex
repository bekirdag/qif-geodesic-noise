% Auto-generated LaTeX from uploaded manuscript + repo materials
% Date: 2026-02-01

\documentclass[11pt]{article}

\usepackage[margin=1in]{geometry}
\usepackage[T1]{fontenc}
\usepackage[utf8]{inputenc}
\usepackage{lmodern}
\usepackage{microtype}

\usepackage{amsmath,amssymb,bm}
\usepackage{graphicx}
\usepackage{booktabs}
\usepackage{enumitem}
\usepackage{hyperref}
\usepackage{xcolor}

\hypersetup{
  colorlinks=true,
  linkcolor=blue,
  citecolor=blue,
  urlcolor=blue
}

\setlist[itemize]{leftmargin=*}
\setlist[enumerate]{leftmargin=*}

\title{A Stochastic-Geodesic-Noise Search for the Einstein Telescope\\
\large A Phenomenological Geodesic-Diffusion Noise Model (formerly ``QIF'')}

\author{Bekir Da\u{g}\\Independent Researcher\\\texttt{bekir@piyote.com}}
\date{}

\begin{document}
\maketitle

\begin{abstract}
We present a \textbf{phenomenological} search framework for \textbf{geodesic diffusion noise} in the Einstein Telescope (ET). [1,3]
The hypothesized signal is a stochastic path-length fluctuation that produces a \textbf{strain power spectral density (PSD) $\propto f^{-\gamma}$}, with \textbf{$\gamma=2$ (random-walk)} as the primary case and other indices as alternatives. [4,5]
The analysis is a nested likelihood-ratio test under a \textbf{complex-Wishart approximation} for Welch cross-spectral estimators, with spline-constrained instrument PSDs and a low-rank environmental coherence model. [2,3,15]
We emphasize that this is \textbf{not} a quantum-gravity derivation: it is an effective stochastic model tested against data and existing experimental limits. [4,6,12]
Significance is calibrated via a parametric bootstrap that is explicitly validated with time-domain simulations. [3,15]
We provide a hardened reference implementation and a commissioning plan, and we show how published experimental bounds map onto the model amplitude. [1,12,13]
\end{abstract}

\noindent\textbf{Keywords:} Einstein Telescope, stochastic noise, cross-spectral density, complex Wishart, Welch estimator, holographic noise, quantum-gravity phenomenology

\section{Scope and limitations}

This manuscript is a \textbf{data-analysis specification} for a \textbf{phenomenological} noise search in ET. [1]
It \textbf{does not} derive quantum gravity. [4,5]
It \textbf{does not} test non-classical gravitational channels (e.g., entanglement-mediated gravity) that are discussed in the quantum-information literature. [17,18,19]
It also does \textbf{not} test post-quantum classical gravity or classical-channel constraints. [24]
It instead defines a classical stochastic noise model and a testable pipeline. [2,3,15]

\noindent\textbf{Explicit limitations:}
\begin{itemize}
  \item The model is \textbf{classical stochastic} and does \textbf{not} address quantum entanglement tests of gravity. [17,18,19]
  \item The complex-Wishart approximation and effective sample size must be \textbf{validated in simulation}. [2,3,15]
  \item The environmental model can mimic signal-like correlations unless constrained with physical priors or witness sensors. [1,3,15]
\end{itemize}

\section{Phenomenological signal model}

\subsection{Generalized geodesic-diffusion spectrum}

We parameterize the one-sided strain PSD as [4,5]
\begin{equation}
S_h(f) = A_h \left(\frac{f}{f_0}\right)^{-\gamma},
\end{equation}
with $\gamma=2$ as the \textbf{random-walk} (Brownian) case and $f_0$ a reference frequency (e.g., 1~Hz). [4,5]
Alternative indices $\gamma\neq 2$ are discussed in the quantum-gravity phenomenology literature. [5,6]
The analysis pipeline can be re-run with different $\gamma$ values if desired. [1,3]

\subsection{Relation to holographic-noise models}

Holographic-noise proposals emphasize \textbf{transverse position indeterminacy} and \textbf{specific correlation structure}, not necessarily a simple $f^{-2}$ random-walk PSD. [7,8,9,10,11]
We therefore treat geodesic diffusion as \textbf{one hypothesis among several} and do \textbf{not} equate it to holographic noise without derivation. [7,8,9,10,11]

\subsection{Amplitude conventions}

We treat $A_h$ as the \textbf{fit parameter}. [4]
If a Planck-scale mapping is desired, it can be written (for $\gamma=2$) as [4,6,14]
\begin{equation}
A_h = \frac{\alpha\,c_0\,\ell_P}{2\pi^2 L^2},
\end{equation}
with $\alpha$ dimensionless. [4,6]
This mapping is \textbf{optional} and must be justified by a specific theoretical model. [4,6]

\section{ET response and covariance structure}

\subsection{Channel response}

Let $\mathbf{x}(f)\in\mathbb{C}^3$ be the three ET strain channels. [1]
The signal covariance is [1,21,22]
\begin{equation}
\bm{\Sigma}_{\mathrm{sig}}(f) = \mathbf{R}(f)\,\mathbf{C}_{\ell}(f)\,\mathbf{R}^{\dagger}(f),
\end{equation}
where $\mathbf{R}(f)$ is the \textbf{ET response + calibration matrix} and $\mathbf{C}_{\ell}(f)$ is the path-length covariance. [1,21,22,6]

\subsection{Idealized covariance template (approximate)}

For equal arms and symmetric path overlaps, one obtains the idealized template. [6,7]
\begin{equation}
\mathbf{M}(\rho)=\begin{pmatrix}
2 & -\rho & -\rho\\
-\rho & 2 & -\rho\\
-\rho & -\rho & 2
\end{pmatrix}.
\end{equation}
This template is \textbf{approximate}; it must be validated against the full ET response. [1,21]
The reference implementation assumes $\mathbf{R}(f)\approx \mathbf{I}$ and uses $\mathbf{M}(\rho)$ as a baseline model. [1,6]

\section{Total covariance model (instrument + environment + signal)}

For each bin $k$, we model the total covariance as follows. [1,2,3]
\begin{equation}
\bm{\Sigma}_k = \mathbf{G}_k\Big(\bm{\Sigma}^{\mathrm{inst}}_k + \bm{\Sigma}^{\mathrm{env}}_k + \bm{\Sigma}^{\mathrm{sig}}_k\Big)\mathbf{G}_k^{\dagger}.
\end{equation}
\begin{itemize}
  \item \textbf{Instrument noise:} diagonal PSDs $P_{ik}$ spline-parameterized in log space. [1,3]
  \item \textbf{Environmental coherence:} low-rank factor model $\mathbf{B}_k\mathbf{B}_k^{\dagger}$ with spline-smoothed coefficients. [3,15]
  \item \textbf{Signal:} $A_h f^{-\gamma}$ mapped through the response matrix. [4,5]
  \item \textbf{Calibration:} phase-only corrections $\phi_{2k},\phi_{3k}$ with channel-1 gauge fixed. [21]
\end{itemize}

\noindent\textbf{Implementation-invariant spline specification:} the reference implementation uses \textbf{open-uniform cubic B-splines} over the analysis band (no extrapolation) with a fixed number of coefficients $N_c$. [1,3]
This ensures reproducibility across implementations. [1,3]

\noindent\textbf{Environmental safeguards:} in commissioning runs, environmental coherence must be constrained using \textbf{line masks}, \textbf{physical priors}, and---where available---\textbf{witness sensors}, to prevent the factor model from absorbing the signal. [1,3,15]

\section{Spectral estimation and normalization}

We use standard Welch estimators with explicit one-sided normalization. [2,3]
\begin{equation}
\widehat{S}_{ij,n}(f)=\frac{2\Delta t}{U}\,\widetilde{x}_{i,n}(f)\,\widetilde{x}^*_{j,n}(f),\quad U=\sum_t w^2[t].
\end{equation}
The bin-averaged cross-spectral matrix $\widehat{\mathbf{S}}_k$ is the average over segments and frequencies within the bin. [2,3]
We \textbf{symmetrize}: [2]
\begin{equation}
\widehat{\mathbf{S}}_k \leftarrow \frac{1}{2}(\widehat{\mathbf{S}}_k+\widehat{\mathbf{S}}_k^{\dagger}).
\end{equation}

\section{Likelihood and inference}

\subsection{Complex-Wishart approximation}

We approximate. [2,3,15]
\begin{equation}
\widehat{\mathbf{S}}_k \sim \mathcal{CW}_3(m_{\mathrm{eff},k},\bm{\Sigma}_k),
\end{equation}
where $m_{\mathrm{eff},k}$ is the effective number of averages (overlap-corrected). [2,3,15]
This is an \textbf{approximation} and must be validated with time-domain simulations. [2,3,15]
Semiclassical gravity analyses emphasize noise kernels and stress-energy fluctuations, supporting a stochastic-noise viewpoint for metric perturbations. [15,16]

\subsection{Non-Gaussian noise handling}

Non-Gaussian features (lines, glitches) can bias the Wishart likelihood. [3,15,20]
The pipeline therefore requires \textbf{line masking}, consistency checks, and (if needed) robust alternatives. [1,3,15]

\section{Significance calibration}

We calibrate the likelihood-ratio statistic with a \textbf{parametric bootstrap}. [3,15]
\begin{enumerate}
  \item Fit $H_{\mathrm{env}}$ to obtain $\widehat{\Theta}_{\mathrm{env}}$. [3,15]
  \item Generate synthetic cross-spectral matrices by sampling complex-Gaussian vectors per bin and forming sample covariances. [3,15]
  The reference implementation uses \textbf{integer-$m$} approximations (round/floor/ceil) with an optional probabilistic rounding mode. [3,15]
  \item Re-fit both hypotheses and compute $\Lambda^{(b)}$. [3,15]
  \item Compute the $p$-value by empirical tail probability. [3,15]
\end{enumerate}

\noindent\textbf{Required validation:} this per-bin bootstrap is approximate and must be validated against time-domain simulations with the actual segmentation and overlap. [3,15]

\section{Validation plan (mandatory)}

\noindent\textbf{Pass/fail criteria (minimum):}
\begin{itemize}
  \item \textbf{False-alarm control:} empirical $p$-value distribution under null must be uniform to within $\pm10\%$ over $p\in[0.1,0.9]$. [1,3]
  \item \textbf{Injection efficiency:} $>90\%$ detection efficiency at a pre-defined $A_h$ threshold with $\gamma=2$, under both rank-1 and rank-2 environmental models. [1,3]
  \item \textbf{Stability:} LR statistic must not shift by more than $10\%$ under alternative knot placements or bootstrap modes. [1,3]
\end{itemize}

\section{Confrontation with existing constraints}

Cosmological stochastic backgrounds (e.g., inflationary models) are distinct components that must be separated from any geodesic-diffusion signal. [2,3,23]
Published experiments already bound spacetime-noise spectra. [12,13,14]
For example, the optical-resonator study reports: [12]
\begin{itemize}
  \item \textbf{Upper limits on normalized distance noise PSD:} $1\times10^{-24}\,\mathrm{Hz^{-1}}$ at 6~mHz and $1\times10^{-28}\,\mathrm{Hz^{-1}}$ above 5~mHz. [12]
  \item \textbf{TAMA 300 reference level:} $S\approx2\times10^{-41}\,\mathrm{Hz^{-1}}$ at $f\sim10^3$~Hz. [12]
  \item \textbf{Random-walk model scale constraint:} $\Lambda>0.6\,\mathrm{nm}$ for the RW2 hypothesis. [12]
\end{itemize}
These bounds can be mapped to $A_h$ via $S_h(f)=S_{\delta\ell}(f)/L^2$. [12,6]
Any claimed detection must be \textbf{consistent} with these constraints and with interferometer-based limits. [12,13,14]

\section{Reference implementation}

The reference implementation in \texttt{qif\_v2.py} and \texttt{qif\_v2\_cuda.py}: [26]
\begin{itemize}
  \item Uses \textbf{open-uniform cubic B-splines} with no extrapolation. [1,3]
  \item Defaults to a \textbf{phenomenological amplitude} $A_h$ and allows Planck-scaled mapping only with an explicit flag. [4,6]
  \item Implements per-bin bootstrap with selectable integer-$m$ rounding modes. [3,15]
\end{itemize}

\noindent\textbf{Code links (public):}
\begin{itemize}
  \item CPU reference implementation (Python):\newline
  \url{https://github.com/bekirdag/qif-geodesic-noise/blob/main/qif_v2.py}
  \item CUDA/GPU accelerated implementation:\newline
  \url{https://github.com/bekirdag/qif-geodesic-noise/blob/main/qif_v2_cuda.py}
\end{itemize}

\subsection{Empirical run notes (ET-MDC1, GPU)}

The ET-MDC1 ``loudest'' sample sets under \texttt{data/} were used (BBH\_snr\_306, BBH\_snr\_344, BBH\_snr\_379, BBH\_snr\_387, BBH\_snr\_587, BNS\_snr\_379), each containing three strain channels (E1/E2/E3) in \texttt{.gwf} files of 2048~s duration. [25]

\paragraph{Small/tuning runs (128~s windows).}
Using \texttt{--max-seconds 128 --max-bins 128 --fit --fit-phi --max-iter 120 --n-starts 2} on GPU, all groups returned the same LR $\approx-4.640351\times10^3$ (bins=127, $r=1$). (See validation logs in the repository. [26])

\paragraph{Full-duration run (2048~s window, downsampled).}
Using \texttt{--max-seconds 2048 --max-bins 512 --fit --fit-phi --max-iter 120 --n-starts 2}, all groups returned the same LR $\approx-3.568421\times10^5$ (bins=512, $r=1$). (See validation logs. [26])

\paragraph{Interpretation.}
These runs do not favor the QIF/geodesic-diffusion term under the tested settings (negative LR). The identical LR across groups suggests that, at the current resolution and model configuration, either (a) the data windows are effectively identical in the relevant statistics, or (b) the fit is not yet sensitive to the injected differences. This is not a detection.

\subsection{Validation suite (bootstrap + injections)}

A bundled validation suite (scripts under \texttt{scripts/} in the repository [26]) was used to check: (i) bootstrap null calibration on real data, and (ii) injection recovery.

\paragraph{Bootstrap on real data.}
A per-bin parametric bootstrap under the fitted environmental model yielded $p\approx1.0$ for the observed negative LR in all groups, consistent with ``no evidence'' for the signal term in this configuration.

\paragraph{Injection recovery (synthetic).}
An injection sweep with a known power-law signal ($\gamma=2$) produced positive LR values increasing with injected amplitude, demonstrating that the optimizer and LR statistic can respond appropriately in a controlled setting.

\subsection{Ubuntu usage (CPU and GPU)}

The repository [26] contains scripts and examples for reproducible execution.
Below is a minimal, tested command sequence (Ubuntu), included for operational reproducibility.

\subsubsection*{GPU (Ubuntu + NVIDIA CUDA)}
From the repository root:
\begin{verbatim}
python3 -m venv .venv
source .venv/bin/activate
pip install -U pip
pip install -r requirements.txt

# Install CuPy (pick the build that matches your CUDA version)
pip install cupy-cuda11x
# or, for CUDA 12.x:
# pip install cupy-cuda12x

# (Optional) sanity check GPU visibility
python -c "import cupy as cp; print('GPU count:', cp.cuda.runtime.getDeviceCount()); print('GPU:', cp.cuda.runtime.getDeviceProperties(0)['name'].decode())"

# Download ET-MDC1 loudest sample data
./scripts/download_data.sh

# Run the GPU pipeline
python qif_v2_cuda.py --data-root data --device auto --max-seconds 128 --max-bins 128 --fit --fit-phi --max-iter 120 --n-starts 2

# Full-duration run (downsampled)
python qif_v2_cuda.py --data-root data --device auto --max-seconds 2048 --max-bins 512 --fit --fit-phi --max-iter 120 --n-starts 2
\end{verbatim}

\subsubsection*{CPU (Ubuntu, no GPU)}
From the repository root:
\begin{verbatim}
python3 -m venv .venv
source .venv/bin/activate
pip install -U pip
pip install -r requirements.txt

# Download ET-MDC1 loudest sample data
./scripts/download_data.sh

# Smoke test (fast)
python qif_v2.py --data-root data --max-seconds 64 --max-bins 64

# Tuned fit (short run)
python qif_v2.py --data-root data --max-seconds 128 --max-bins 128 --fit --fit-phi --max-iter 120 --n-starts 2

# Full-duration run (downsampled)
python qif_v2.py --data-root data --max-seconds 2048 --max-bins 512 --fit --fit-phi --max-iter 120 --n-starts 2
\end{verbatim}

\subsubsection*{Notes}
\begin{itemize}
  \item The download script uses \texttt{BASE\_URL=http://et-origin.cism.ucl.ac.be/} by default.
  \item If your system CUDA version is unknown, run \texttt{nvidia-smi}.
\end{itemize}

\subsubsection*{Validation scripts}
From the repository root:
\begin{verbatim}
# Sensitivity sweep (resolution test)
./scripts/run_sensitivity_sweep.sh

# Bootstrap p-values
./scripts/run_bootstrap.sh

# Stress tests (rank2 + calib variants)
./scripts/run_stress_tests.sh

# Line mask + transfer function
./scripts/run_line_mask_transfer.sh

# Synthetic injection recovery
python scripts/run_injection_test.py
\end{verbatim}

\section*{References}

\begin{enumerate}[label={[\arabic*]}]

\item Data Analysis Challenges for the Einstein Telescope (arXiv:0910.0380).\newline
\url{https://arxiv.org/abs/0910.0380}

\item The stochastic gravity-wave background: sources and detection (arXiv:gr-qc/9604033).\newline
\url{https://arxiv.org/abs/gr-qc/9604033}

\item Stochastic Gravitational-Wave Backgrounds: Current Detection Efforts and Future Prospects (arXiv:2107.00129).\newline
\url{https://arxiv.org/abs/2107.00129}

\item A phenomenological description of space-time noise in quantum gravity (arXiv:gr-qc/0104086).\newline
\url{https://arxiv.org/abs/gr-qc/0104086}

\item Quantum foam and quantum gravity phenomenology (arXiv:gr-qc/0405078).\newline
\url{https://arxiv.org/abs/gr-qc/0405078}

\item Gravity-wave interferometers as probes of a low-energy effective quantum gravity (arXiv:gr-qc/9903080).\newline
\url{https://arxiv.org/abs/gr-qc/9903080}

\item Holographic Noise in Interferometers (arXiv:0905.4803).\newline
\url{https://arxiv.org/abs/0905.4803}

\item Measurement of Quantum Fluctuations in Geometry (arXiv:0712.3419).\newline
\url{https://arxiv.org/abs/0712.3419}

\item Indeterminacy of Holographic Quantum Geometry (arXiv:0806.0665).\newline
\url{https://arxiv.org/abs/0806.0665}

\item Holographic Indeterminacy, Uncertainty and Noise (arXiv:0709.0611).\newline
\url{https://arxiv.org/abs/0709.0611}

\item Spacetime Indeterminacy and Holographic Noise (arXiv:0706.1999).\newline
\url{https://arxiv.org/abs/0706.1999}

\item Experimental limits for low-frequency space-time fluctuations from ultrastable optical resonators (arXiv:gr-qc/0401103).\newline
\url{https://arxiv.org/abs/gr-qc/0401103}

\item The GEO 600 gravitational wave detector (DOI:10.1088/0264-9381/19/7/321).\newline
\url{https://doi.org/10.1088/0264-9381/19/7/321}

\item Quantum gravity motivated Lorentz symmetry tests with laser interferometers (arXiv:gr-qc/0306019).\newline
\url{https://arxiv.org/abs/gr-qc/0306019}

\item Stochastic Gravity: Theory and Applications (arXiv:0709.4457).\newline
\url{https://arxiv.org/abs/0709.4457}

\item Noise and Fluctuations in Semiclassical Gravity (arXiv:gr-qc/9312036).\newline
\url{https://arxiv.org/abs/gr-qc/9312036}

\item Locality and entanglement in table-top testing of the quantum nature of linearized gravity (arXiv:1801.02708).\newline
\url{https://arxiv.org/abs/1801.02708}

\item Gravity is not a Pairwise Local Classical Channel (arXiv:1612.07735).\newline
\url{https://arxiv.org/abs/1612.07735}

\item Gravitationally Mediated Entanglement: Newtonian Field vs. Gravitons (arXiv:2112.10798).\newline
\url{https://arxiv.org/abs/2112.10798}

\item Testing Quantum Gravity by Quantum Light (arXiv:1304.7912).\newline
\url{https://arxiv.org/abs/1304.7912}

\item Quantum interactions between a laser interferometer and gravitational waves (DOI:10.1103/PhysRevD.98.124006).\newline
\url{https://doi.org/10.1103/PhysRevD.98.124006}

\item LISA Laser Interferometer Space Antenna for gravitational wave measurements (DOI:10.1088/0264-9381/13/11A/033).\newline
\url{https://doi.org/10.1088/0264-9381/13/11A/033}

\item Stochastic Gravity Wave Background in Inflationary Universe Models (DOI:10.1103/PhysRevD.37.2078).\newline
\url{https://doi.org/10.1103/PhysRevD.37.2078}

\item The constraints of post-quantum classical gravity (arXiv:1707.06050).\newline
\url{https://arxiv.org/abs/1707.06050}

\item Mock data challenge for the Einstein Gravitational-Wave Telescope (Phys. Rev. D 86, 122001).\newline
\url{https://doi.org/10.1103/PhysRevD.86.122001}

\item Code repository, scripts, and validation logs (GitHub).\newline
\url{https://github.com/bekirdag/qif-geodesic-noise}

\end{enumerate}

\end{document}
